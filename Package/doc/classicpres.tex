%%%%%%%%%%%%%%%%%%%%%%%%%%%%%%%%%%%%%%%%%%%%%%%%%%%%%%%%%%%%%%%%%%%%%%%%%%%%
%
%A  classicpres.tex  
%
%
%

\Chapter{The classicpres package}

This package provides functionality for presentations of classical groups
as given in \cite{lgo20}. It is a translation (with permission of the
authors) of the corresponding code in the system
Magma, and should -- short of arbitrary choices depending on internal
ordering of finite field elements -- return the same data.
\medskip

Work on the translation was started at the "Summer School
Matrix Group Recognition" at RWTH Aachen in July 2019.

%%%%%%%%%%%%%%%%%%%%%%%%%%%%%%%%%%%%%%%%%%%%%%%%%%%%%%%%%%%%%%%%%%%%%%%%%
\Section{Provided Functions}

There are only two user functions:

\>ClassicalStandardGenerators(<type>, <d>, <q>)

This operation produces the standard generators of Leedham-Green and O'Brien
for the quasisimple classical group of specified type in dimension <d> over a
field of size <q>. The type is designated by the argument type which must be
one of the strings `"SL"', `"Sp"', `"SU"', `"Omega"', `"Omega-"', or
`"Omega+"'. The
standard generators generate a specific copy of a classical group and are
defined in \cite{LGO09} and \cite{DLLGO13}.

\>ClassicalStandardPresentation(<type>, <d>, <q> )

Given the specification <type>, <d>, <q> of a quasisimple group $G$, this
operation constructs a presentation on the standard generators for $G$.
The string <type> must be one of `"SL"', `"Sp"', `"SU"', `"Omega"',
`"Omega-"', or `"Omega+"', while <d> is the dimension and <q> is the
cardinality of the finite field. The presentations are described in
\cite{LGO20}.
The presentation is returned as a finitely presented group, the relations
being stored as its `RelatorsOfFpGroup'

If the option `Projective' is set to true, the operation constructs a
presentation for the corresponding projective group (on the images of the
same generators).

%%%%%%%%%%%%%%%%%%%%%%%%%%%%%%%%%%%%%%%%%%%%%%%%%%%%%%%%%%%%%%%%%%%%%%%%%
\Section{Isomorphism to finitely presented group}

In {\GAP}, the canonical way of obtaining a presentation of (say) a
permutation group is the operation `IsomorphismFpGroup'. This package thus
also installs a method for `IsomorphismFpGroup' for groups that know that
they are simple, and tests whether the groups are classical. If so, an
isomorphism between the given group, and the permutation action of the
classical groups on vectors is computed (this should be improved with
constructive recognition in the future) and the presentation of
`ClassicalStandardPresentation' is used.

Non-simple groups that construct a presentation through their composition
factors will then automatically use these presentations for the simple
factors when combining to a presentation of the group.

However be aware that the isomorphisms returned will use the generic
permutation group mechanism for decomposition, and thus will not result in
good or short words.

%%%%%%%%%%%%%%%%%%%%%%%%%%%%%%%%%%%%%%%%%%%%%%%%%%%%%%%%%%%%%%%%%%%%%%%%%
\Section{Standard Generators}

The generating sequences chosen (and on which the presentations are written)
use the standard generators of classical groups as defined in
\cite{LGO09} and \cite{DLLGO13}, since these generators also are used in the
constructive recognition process.

The presentations in \cite{LGO20} initially use slightly different generating
sets, and a translation betweenthese two generating sets is provided. If one
of the two functions `ClassicalStandardGenerators' or
`ClassicalStandardPresentation' is called with the option
`PresentationGenerators', this different generating set is used. (It's use
could be to obtain a slightly shorter presentation in situations in which
adherence to the standard generaors is not required.)

The method provided for `IsomorphismFpGroup' currently uses these
`PresentationGenerators', but this is not a guaranteed property.


